\section{Evaluation und Optimierung}

Die Validierung der implementierten Suchfunktionalitäten erfolgte mittels eines automatisierten Test-Scripts mit 48 verschiedenen Suchszenarien. Alle Tests liefen erfolgreich durch (Success Rate: 100\%), was zunächst einmal die technische Stabilität des Systems bestätigt.

\subsection{Funktionale Tests der Suchanfragetypen}

\subsubsection{Namensbasierte Suche}

Die exakte Namenssuche funktioniert tadellos -- bei Queries wie \texttt{pikachu} oder \texttt{charizard} erreicht das System perfekte Werte (Precision, Recall und F-Measure jeweils 1.0). Interessant wird es bei der partiellen Namenssuche mit Wildcards: Hier zeigt sich, dass Substring-Matching grundsätzlich gut funktioniert, wobei kürzere Fragmente wie \texttt{char} erwartungsgemäß mehr (und damit weniger präzise) Treffer liefern als spezifischere wie \texttt{saur}.

\begin{table}[h!]
\centering
\begin{tabular}{|l|c|c|c|}
\hline
\textbf{Suchtyp} & \textbf{Precision} & \textbf{Recall} & \textbf{F-Measure} \\
\hline
Exakte Namen & 1.00 & 1.00 & 1.00 \\
Partielle Namen & 0.71 & 1.00 & 0.80 \\
\hline
\end{tabular}
\caption{Performance der namensbasierten Suche}
\end{table}

\subsubsection{Thematische Suche}

Hier wird es problematisch: Die typ-basierte Suche ist praktisch defekt -- Suchanfragen nach \texttt{fire}, \texttt{water} oder \texttt{electric} liefern entweder gar keine oder völlig irrelevante Ergebnisse. Das ist ein kritischer Bug, der dringend behoben werden muss.

Die Fähigkeiten-Suche funktioniert zwar technisch, hat aber ein massives Präzisionsproblem: Bei Queries wie \texttt{overgrow} oder \texttt{blaze} werden zwar alle relevanten Pokémon gefunden (Recall = 1.0), aber gleichzeitig auch jede Menge irrelevante Treffer, was zu einer miserablen Precision von nur 0.15 führt.

\subsubsection{Interaktive Features}

Das Autocomplete arbeitet blitzschnell (8-9ms Antwortzeit) und liefert zu 100\% relevante Vorschläge -- hier läuft alles rund. Die Rechtschreibkorrektur hingegen existiert zwar, macht aber praktisch nichts: Tippfehler werden nicht korrigiert, obwohl das System nicht abstürzt.

\subsection{Information Retrieval Metriken}

Die Gesamtperformance des Systems zeigt ein gemischtes Bild:

\begin{table}[h!]
\centering
\begin{tabular}{|l|c|l|}
\hline
\textbf{Metrik} & \textbf{Wert} & \textbf{Bewertung} \\
\hline
Overall Precision & 0.44 & Ausbaufähig \\
Overall Recall & 0.93 & Sehr gut \\
F-Measure & 0.52 & Okay \\
Mean Reciprocal Rank & 1.00 & Perfekt \\
Mean Average Precision & 0.90 & Sehr gut \\
\hline
\end{tabular}
\caption{Gesamtperformance des Suchsystems}
\end{table}

Das System hat eine interessante Charakteristik: Es findet fast alles was relevant ist (hoher Recall), produziert dabei aber viel "Rauschen" in Form irrelevanter Treffer (niedrige Precision). Positiv ist, dass das erste Ergebnis praktisch immer das richtige ist (MRR = 1.0).

\subsection{Performance-Analyse}

Mit durchschnittlich 16ms Antwortzeit ist das System erfreulich schnell -- alle Anfragen werden in unter 20ms beantwortet. Hier gibt es definitiv keine Performance-Probleme.

\subsection{Identifizierte Optimierungsfelder}

Basierend auf den quantitativen Ergebnissen lassen sich klare Prioritäten ableiten:

\textbf{Identifizierte Verbesserungsfelder:}

\textbf{Kritische Probleme:}
\begin{itemize}
\item Typ-basierte Suche ist komplett defekt und liefert keine verwertbaren Ergebnisse
\item Overall Precision von 0.44 deutlich unter dem angestrebten Niveau von >0.6
\end{itemize}

\textbf{Relevante Schwächen:}
\begin{itemize}
\item Fähigkeiten-Suche hat mit 15\% Precision erhebliche Relevanzprobleme
\item Ranking-Qualität in den Top-10 Ergebnissen ist verbesserungswürdig
\end{itemize}

\textbf{Optimierungspotenzial:}
\begin{itemize}
\item Query-Expansion könnte thematische Suchen verbessern
\item Rechtschreibkorrektur funktioniert grundsätzlich, könnte aber erweitert werden
\end{itemize}

\subsection{Fazit der Evaluation}

Das entwickelte Suchsystem zeigt eine klare Zweiteilung in der Funktionalität: Während namensbasierte Suchen zufriedenstellende Ergebnisse liefern und durch gute Performance (16ms Antwortzeit) überzeugen, offenbart die systematische Evaluation mit standardisierten IR-Metriken erhebliche Defizite bei thematischen Suchfunktionen. 

Die dokumentierten Schwächen -- insbesondere die defekte Typ-Suche und die niedrige Overall Precision -- zeigen konkrete Ansatzpunkte für zukünftige Weiterentwicklungen auf. Die 100\%ige technische Stabilität und die funktionale Grundarchitektur demonstrieren die erfolgreiche Umsetzung der IR-Konzepte und bieten eine solide Basis für entsprechende Verbesserungen.