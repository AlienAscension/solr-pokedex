\section{Evaluation und Optimierung}

\subsection{Funktionale Tests der implementierten Suchanfragetypen}

Die Validierung der Suchfunktionalitäten erfolgte durch systematische Tests mit einem automatisierten Test-Script, das 39 verschiedene Suchszenarien gegen die laufende Anwendung ausführte. Alle Tests wurden erfolgreich abgeschlossen (Success Rate: 100\%), was die technische Stabilität des Systems bestätigt.

\subsubsection{Namensbasierte Suche}

\textbf{Exakte Namenssuche:} Tests mit vollständigen Pokemon-Namen wie "`Pikachu"', "`Charizard"' und "`Bulbasaur"' ergaben eine nahezu perfekte Trefferquote. Fünf von sechs getesteten Pokemon standen an Position 1, während "`Mew"' an Position 2 erschien, vermutlich aufgrund der Kürze des Namens und möglicher Übereinstimmungen mit anderen Feldern.

\textbf{Partielle Namenssuche:} Die Wildcard-Funktionalität zeigte hervorragende Ergebnisse bei Substring-Matching. Eingaben wie "`pika"' (1 Treffer), "`char"' (7 Treffer) und "`saur"' (3 Treffer) demonstrieren die korrekte Implementierung der \texttt{name:*query*}-Pattern-Suche. Die Suchanfrage "`char"' fand erfolgreich alle Charmander-Evolutionslinien-Pokemon.

\subsubsection{Thematische Suche-Evaluation}

\textbf{Typ-basierte Suche:} Die Tests zeigten gemischte Ergebnisse für Typ-Suchen. Während "`fire"' ein Ergebnis lieferte, ergaben "`water"', "`grass"', "`electric"' und "`psychic"' keine Treffer. Dies deutet darauf hin, dass die Field-Boosting-Konfiguration für Typ-Suchen optimiert werden könnte, oder dass die getesteten Begriffe nicht exakt mit den indexierten Typ-Werten übereinstimmen.

\textbf{Fähigkeiten-Suche:} Alle getesteten Fähigkeiten ("`overgrow"', "`blaze"', "`torrent"', "`static"') ergaben null Treffer. Eine Analyse der Ursache zeigte, dass die Fähigkeiten-Namen in der Datenbank möglicherweise in formatierter Form (z.B. "`Overgrow"' statt "`overgrow"') gespeichert sind, was Case-Sensitivity-Probleme verursacht.

\subsubsection{Interaktive Features}

\textbf{Autocomplete-Performance:} Die Autocomplete-Funktionalität zeigte exzellente Ergebnisse mit durchschnittlichen Antwortzeiten von 8-9ms und einer Perfect Relevance Rate von 100\% für alle getesteten Eingaben. Die Anzahl der Vorschläge variierte sinnvoll: "`pika"' (1 Vorschlag), "`char"' (7 Vorschläge), "`bulb"' (2 Vorschläge), "`saur"' (3 Vorschläge).

\textbf{Rechtschreibkorrektur:} Die SpellCheck-Tests ergaben, dass das System zwar technisch funktioniert (keine Fehler), aber keine Korrekturvorschläge für die getesteten Tippfehler "`piakchu"', "`charizrd"', "`bulbasr"' und "`squirtl"' lieferte. Dies deutet darauf hin, dass das SpellCheck-Dictionary möglicherweise nicht vollständig aufgebaut wurde oder die Fehlertoleranz-Schwellenwerte zu restriktiv sind.

\subsubsection{Filter-Funktionalität}

Die facettierte Suche funktionierte zuverlässig mit realistischen Ergebniszahlen:
\begin{itemize}
    \item Generation 1 Filter: 790 Treffer
    \item Fire Type Filter: 81 Treffer  
    \item Legendary Filter: 94 Treffer
    \item Kombinierte Filter (Gen 1 + Fire): 65 Treffer
    \item Kombinierte Filter (Gen 1 + Legendary): 78 Treffer
\end{itemize}

Die Kombinationslogik funktioniert korrekt, da die kombinierten Filter weniger Ergebnisse liefern als die Einzelfilter.

\subsection{Performance-Analyse}

\subsubsection{Antwortzeit-Metriken}

Das System zeigte hervorragende Performance-Werte:
\begin{itemize}
    \item \textbf{Durchschnittliche Antwortzeit:} 12ms
    \item \textbf{Schnellste Antwort:} 8ms (einfache Queries)
    \item \textbf{Langsamste Antwort:} 30ms (sehr lange Query)
    \item \textbf{Autocomplete-Performance:} 8-9ms durchschnittlich
\end{itemize}

Diese Werte liegen deutlich unter der 100ms-Schwelle für gefühlte Echtzeit-Interaktion und gewährleisten eine flüssige Benutzererfahrung.

\subsubsection{Skalierbarkeits-Indikatoren}

Die Edge-Case-Tests zeigten robustes Verhalten:
\begin{itemize}
    \item Leere Queries werden korrekt behandelt (1025 Gesamtergebnisse)
    \item Sehr lange Queries (30ms Antwortzeit) degradieren graceful
    \item Sonderzeichen werden sicher verarbeitet (0 Treffer, keine Fehler)
    \item Single-Character-Suchen sind funktional (896 Treffer für "`a"')
\end{itemize}

\subsection{Bewertung und Optimierung des Relevanzrankings}

\subsubsection{Relevanz-Metriken}

Die Relevanz-Bewertung ergab:
\begin{itemize}
    \item \textbf{Durchschnittlicher Relevanz-Score:} 0.47
    \item \textbf{Perfect Relevance Rate:} 46.2\%
    \item \textbf{Success Rate:} 100\% (technische Funktionalität)
\end{itemize}

Der moderate Relevanz-Score von 0.47 deutet auf Optimierungspotential beim Field-Boosting hin, insbesondere für thematische Suchen.

\subsubsection{Identifizierte Optimierungsansätze}

Basierend auf den Test-Ergebnissen wurden folgende Verbesserungsmöglichkeiten identifiziert:

\textbf{Case-Insensitive Suche:} Die Implementierung einer case-insensitive Suche für Fähigkeiten und Typen könnte die Trefferquote erheblich verbessern.

\textbf{SpellCheck-Konfiguration:} Die Rechtschreibkorrektur benötigt eine Anpassung der Toleranz-Schwellenwerte oder eine vollständige Neuerstellung des Dictionaries.

\textbf{Field-Boosting-Anpassung:} Eine Erhöhung der Gewichtung für \texttt{types} und \texttt{all\_abilities} könnte thematische Suchen verbessern.

\textbf{Query-Expansion:} Die Implementierung von Synonymen (z.B. "`fire"' → "`fire type"') könnte die Trefferquote für Typ-Suchen erhöhen.

\subsection{Quantitative Evaluation der Suchergebnisse}

Die systematische Evaluation mit 39 Test-Cases lieferte messbare Qualitätsindikatoren für die verschiedenen Suchfunktionalitäten. Das implementierte System erreicht exzellente Performance-Werte bei Namens-basierten Suchen und Autocomplete-Funktionalität, zeigt jedoch Verbesserungspotential bei thematischen Suchen und Rechtschreibkorrektur. Die 100\%ige technische Success Rate bestätigt die Robustheit der Implementierung.