\section{Fazit und Ausblick}
\label{chap:fazit}

\subsection{Zusammenfassung der Projektergebnisse}
\label{sec:fazit_zusammenfassung}
Fassen Sie die erreichten Ziele und das finale Produkt noch einmal kurz zusammen.

\subsection{Reflektion der Herausforderungen und Lösungsansätze}
\label{sec:fazit_herausforderungen}
Was waren die größten Schwierigkeiten im Projekt (z.B. Schema-Design, Daten-Parsing, Flask-Anbindung) und wie haben Sie diese gelöst?

\subsection{Mögliche Erweiterungen und zukünftige Optimierungen}
\label{sec:fazit_ausblick}
Welche Ideen haben Sie für die Zukunft? (z.B. weitere Datenquellen, komplexere Filter, Benutzer-Accounts, etc.)


% --- Literaturverzeichnis ---
% Falls Sie BibTeX/BibLaTeX verwenden, kommt hier der Befehl zum Drucken hin.
% Beispiel für BibLaTeX:
% \printbibliography[heading=bibintoc, title={Literatur- und Quellenverzeichnis}]


% --- Anhang ---
\appendix % Schaltet auf Anhangs-Modus (A, B, C...)
\section{Auszug aus dem Solr-Schema}
\label{app:solr_schema}
Fügen Sie hier relevante Teile Ihrer `managed-schema` ein, z.B. als Code-Block.

\section{Relevante Code-Auszüge}
\label{app:code}
Zeigen Sie hier wichtige Snippets aus \texttt{fetcher\_v2.py} oder \texttt{web\_app.py}.

\section{Screenshot der Benutzeroberfläche}
\label{app:screenshot}
Ein vollseitiger Screenshot der fertigen Anwendung.