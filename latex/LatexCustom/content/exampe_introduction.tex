\section{Introduction}
\lipsum[1]

\subsection{Bold, Italic, Underlining}
\textbf{Bold}, \textit{italic}, \underline{underline}, \textbf{\textit{\underline{nested}}}.

\subsection{Figures}
As in images or pictures within your document.

\begin{figure}[h!]
    \centering
    \includegraphics[width=0.4\linewidth]{ch01_osaka_dotonbori.jpg}
    \caption{Dotonbori district, Osaka, Japan}
    \label{fig:dotonbori}
\end{figure}

Figure \ref{fig:dotonbori} shows Dotonbori district in Osaka, Japan.

\subsubsection{Multiple Figures}

Here you see multiple figures side by side

\begin{figure}[h!]
  \centering
  \begin{subfigure}[b]{0.4\linewidth}
    \includegraphics[width=\linewidth]{ch01_osaka_dotonbori.jpg}
    \caption{Dotonbori District.}
    \label{fig:osaka-dotonbori-district}
  \end{subfigure}
  \begin{subfigure}[b]{0.4\linewidth}
    \includegraphics[width=\linewidth]{ch01_takoyaki.jpeg}
    \caption{Takoyaki.}
    \label{fig:fresh-takoyaki}
  \end{subfigure}
  \caption{two things you can find in Japan.}
  \label{fig:twothingsinjpn}
\end{figure}

Osaka is a beautiful city with many delights (\ref{fig:twothingsinjpn}).
Such as Dotonbori district shown in figure \ref{fig:osaka-dotonbori-district}, where you can get delicious Takoyaki seen in figure \ref{fig:fresh-takoyaki}.

\subsection{Tables}

\subsubsection{Aligned Table}

\begin{table}[h!]
  \begin{center}
    \begin{tabular}{l|S|r} % <-- Changed to S here.
      \textbf{Value 1} & \textbf{Value 2} & \textbf{Value 3}\\
      $\alpha$ & $\beta$ & $\gamma$ \\
      \hline
      1 & 1110.1 & a\\
      2 & 10.1 & b\\
      3 & 23.113231 & c\\
    \end{tabular}
  \end{center}
  \caption{Table with aligned units.}
  \label{tab:table1}
\end{table}

This is an amazing table and you reference it like \ref{tab:table1}.

\subsubsection{Table with lots of text}

\begin{table}[htbp!]
  \begin{center}
    \begin{tabular}{|p{5cm}|p{5cm}|p{5cm}|}
	  \hline
	  \textbf{Column 1} & \textbf{Column 2} & \textbf{Column 3} \\
	  \hline
	  This is a longer sentence that will automatically wrap within the cell due to the p{5cm} column specification. & Another lengthy sentence that demonstrates how text wrapping works in LaTeX tables when the content is too long. & The third column also supports multiple lines of text that will wrap automatically. \\
	  \hline
	  Short text. & Medium length sentence here. & Another wrapped sentence that continues onto the next line automatically. \\
	  \hline
    \end{tabular}
  \end{center}
  \caption{Table with wrapped text}
  \label{tab:wrapped}
\end{table}

\subsection{Lists}

\subsubsection{Unordered List}
\begin{itemize}
    \item One
    \item Two
    \item Three
\end{itemize}

\subsubsection{Numbered List}

\begin{enumerate}
    \item One
    \item Two
    \item Three
\end{enumerate}

\subsection{Sourcecode}

This is how you can show Code Snippets:
% YOU MIGHT NEED THIS NEWLINE HERE

\noindent
\begin{minipage}{\linewidth}
\lstinputlisting[language=Go, caption=Simple Go Program]{sourcecode/hello.go}
\end{minipage}

\subsection{Citations}
We're using \textit{BibLaTeX}, which is mostly compatible with \textit{BibTeX}\cite{biblatex-bibtex-compatibility}.

\subsubsection{Basic Citation Commands}

\paragraph{Standard Numeric Citation:}
Version control systems are essential for software development\cite{git}.
\\

\paragraph{Author as Part of Sentence:}
\textcite{git} revolutionized distributed version control when he created Git in 2005.
\\

\paragraph{Parenthetical Citation:}
Modern web frameworks have made Go development more accessible \parencite{gin-framework}.
\\

\paragraph{Author Only:}
The design principles described by \citeauthor{golang-docs} emphasize simplicity and readability.
\\

\paragraph{Title Only:}
For comprehensive documentation, see \citetitle{golang-docs}.
\\

\subsubsection{Advanced Citation Features}

\paragraph{Multiple Citations:}
Several tools and frameworks support modern development workflows\cite{git,golang-docs,gin-framework}.
\\

\paragraph{Page-Specific Citations:}
The installation instructions are clearly documented\cite[S.~15]{golang-docs}.
\\

\paragraph{Citations with Prenotes and Postnotes:}
For a detailed comparison of version control systems\cite[see][S.~42--45]{git}.
\\

\paragraph{Prenotes Only:}
\cite[cf.][]{DUMMY:1}
\\

\paragraph{Range of Pages:}
The framework architecture is explained in detail\cite[S.~10--25]{gin-framework}.
\\

\subsubsection{Different Entry Types Demonstration}

\paragraph{Book Reference:}
Classical programming concepts are covered in \textcite{DUMMY:1}.

\paragraph{Journal Article:}
Recent advances in the field are documented\cite{ARTICLE:1}.

\paragraph{Website/Online Resource:}
Corporate documentation can be found online\cite{WEBSITE:1}.

\paragraph{Book Chapter:}
Specific implementation details are covered\cite[Chapter~5]{BOOK:2}.

\subsubsection{Bibliography Styles}
Check the various entry types and their formatting in the \textit{references.bib} file. BibLaTeX automatically handles different source types (books, articles, websites, software) and formats them according to the chosen style (IEEE in this case).

\subsection{Footnotes}
I really like Yakitori\footnote{\label{fn:fishwarning}Contains fish.} fresh from the grill-thingy.
Keep Footnote \ref{fn:fishwarning} in mind.
Although referencing footnotes is questionable.
Single word footnotes have no punctuation.

\subsection{The Glossary \& Acronyms}
Note that often times it can be helpful to start with a glossary-entry when first mentioning an acronym, \acrshort{eg} an \Gls{api-gls}.
The \Gls{latex} typesetting markup language is specially suitable for documents that include \gls{maths}.
\Glspl{formula} are rendered properly an easily once one gets used to the commands.

\subsubsection{German Plural}
Ein Computer ist ein sehr nützliches Gerät für die Arbeit mit \Glspl{datei}.

\subsubsection{Acronyms}
Given a set of numbers, there are elementary methods to compute its \acrlong{gcd}, which is abbreviated \acrshort{gcd}.
This process is similar to that used for the \acrfull{lcm}.
For fast prototyping it is recommended to first get \acrfullpl{mwe} going, before caring about the porcelain.
