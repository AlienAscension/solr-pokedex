% Preamble
% ---
\documentclass{article}

% Set margins
% ---
\usepackage[
  left=2cm,
  right=2cm,
  top=2cm,
  bottom=2cm
]{geometry}

% make sections start on a new page
% ---
\usepackage{titlesec}
\newcommand{\sectionbreak}{\clearpage}

% Packages
% ---
\usepackage[utf8]{inputenc} % Unicode support (Umlauts etc.)
\usepackage[ngerman]{babel} % Change hyphenation rules
\usepackage{microtype} % Improves typography and reduces overfull boxes
\usepackage{graphicx} % Add pictures to your document
\usepackage{listings} % Source code formatting and highlighting
\usepackage{graphicx} % Figures: enable images
\graphicspath{{figures/}} % Figures: path to figures
\usepackage{subcaption} % allow for figures side by side
\usepackage{siunitx} % aligns table columns / rows
\sisetup{
  round-mode          = places, % Rounds numbers
  round-precision     = 2, % to 2 places
} % table cell value rounding
\usepackage{listings} % Source Code formatting
\usepackage{xcolor}
\usepackage{pgf-pie} % For Pie Charts
\usepackage{pgfplots}

% BibLaTeX Setup
% ---
\usepackage[
    backend=biber,          % Use biber instead of bibtex
    style=ieee,             % IEEE style (equivalent to IEEEtran)
    sorting=nyt,            % Sort by name, year, title
    citestyle=numeric,      % Numeric citations [1], [2], etc.
    bibstyle=numeric,       % Numeric bibliography
    maxbibnames=10,         % Maximum authors shown in bibliography
    maxcitenames=2,         % Maximum authors shown in citations
    giveninits=true,        % Use initials for first names
    uniquename=init,        % Handle name disambiguation
    uniquelist=false,       % Don't use unique lists
    hyperref=auto,          % Enable hyperlinks if hyperref is loaded
    backref=false,          % Don't show back-references
]{biblatex}

% Add your bibliography file
\addbibresource{references.bib}  % Note: .bib extension required

% Define Rose Pine colors
\definecolor{rosebase}{RGB}{250,244,237}
\definecolor{rosesurface}{RGB}{255,250,243}
\definecolor{roseoverlay}{RGB}{242,233,222}
\definecolor{rosemuted}{RGB}{152,147,165}
\definecolor{rosesubtle}{RGB}{121,117,147}
\definecolor{rosetext}{RGB}{87,82,121}
\definecolor{roselove}{RGB}{180,99,122}
\definecolor{rosegold}{RGB}{234,157,52}
\definecolor{roserose}{RGB}{215,130,126}
\definecolor{rosepine}{RGB}{40,105,131}
\definecolor{rosefoam}{RGB}{86,148,159}
\definecolor{roseiris}{RGB}{144,122,169}

\lstdefinestyle{gostyle}{ % General setup for the package
    language=Go,
    backgroundcolor=\color{rosebase},
    rulecolor=\color{rosebase},
    rulesepcolor=\color{rosebase},
    commentstyle=\color{rosemuted}\itshape,
    keywordstyle=\color{roseiris}\bfseries,
    numberstyle=\tiny\color{rosemuted},
    stringstyle=\color{rosegold},
    basicstyle=\small\sffamily\color{rosetext},
    breakatwhitespace=false,
    breaklines=true,
    captionpos=b,
    keepspaces=true,
    numbers=left,
    numbersep=5pt,
    showspaces=false,
    showstringspaces=false,
    showtabs=false,
    tabsize=4,
    numberstyle=\tiny\color{rosemuted},
    frame=none,
    columns=fixed,
    % Syntax highlighting colors
    identifierstyle=\color{rosetext},
    % Additional Go-specific keywords
    morekeywords={func, var, const, type, struct, interface, map, chan, go, defer, select, fallthrough, range, package, import},
    % Additional styling
    emphstyle=\color{rosefoam}
}

\lstdefinelanguage{json}{
    basicstyle=\footnotesize\ttfamily,
    numbers=left,
    numberstyle=\scriptsize,
    stepnumber=1,
    numbersep=8pt,
    showstringspaces=false,
    breaklines=true,
    frame=lines,
    backgroundcolor=\color{gray!10},
    literate=
     *{0}{{{\color{blue}0}}}{1}
      {1}{{{\color{blue}1}}}{1}
      {2}{{{\color{blue}2}}}{1}
      {3}{{{\color{blue}3}}}{1}
      {4}{{{\color{blue}4}}}{1}
      {5}{{{\color{blue}5}}}{1}
      {6}{{{\color{blue}6}}}{1}
      {7}{{{\color{blue}7}}}{1}
      {8}{{{\color{blue}8}}}{1}
      {9}{{{\color{blue}9}}}{1}
      {:}{{{\color{red}{:}}}}{1}
      {,}{{{\color{red}{,}}}}{1}
      {\{}{{{\color{red}{\{}}}}{1}
      {\}}{{{\color{red}{\}}}}}{1}
      {[}{{{\color{red}{[}}}}{1}
      {]}{{{\color{red}{]}}}}{1},
    string=[s]{"}{"},
    stringstyle=\color{green!60!black},
    comment=[l]{//},
    commentstyle=\color{gray},
}

% set initial style
\lstset{style=gostyle}

\usepackage[nottoc,notlot,notlof]{tocbibind} % Include bibliography in TOC % EDIT THIS
\usepackage[acronym, toc]{glossaries}
\makeglossaries
% A

\newglossaryentry{api-gls}
{
        name={Application Programming Interface},
        description={
                A set of rules, protocols, and tools that allows different software applications to communicate with each other.
                APIs define the methods and data formats that applications can use to request and exchange information, enabling integration between different systems without requiring knowledge of their internal implementation.
        }
}

% D

\newglossaryentry{datei}
{
        name={Datei},
        description={Eine strukturierte Sammlung von Informationen},
        plural={Dateien}
}

% F

\newglossaryentry{formula}
{
        name=formula,
        description={A mathematical expression}
}

% L

\newglossaryentry{latex}
{
        name=latex,
        description={Is a mark up language specially suited for scientific documents}
}

% M

\newglossaryentry{maths}
{
        name=mathematics,
        description={Mathematics is what mathematicians do}
}

% A
\newacronym{api}{API}{Application Programming Interface}

% E
\newacronym{eg}{e.g.}{exempli gratia}

% G
\newacronym{gcd}{GCD}{Greatest Common Divisor}

% L
\newacronym{lcm}{LCM}{Least Common Multiple}

% M
\newacronym[shortplural={MWEs}, longplural={minimal working examples}]{mwe}{MWE}{minimal working example}

%\input{ads/appendix-figures}

% used for Templating
% ---
\usepackage{lipsum}

% Table Of Contents
\setcounter{tocdepth}{3} % show subsections in toc

% Document
% ---
\begin{document}
	\pagenumbering{gobble} % no pagenumbers

	% Custom Title Page
\begin{titlepage}
    \begin{center}
        \vspace*{2cm}

        \Huge
        \textbf{Information Retrieval – Dokumentation}

        \vspace{0.5cm}
        \LARGE
        bei Sascha Szott

        \vspace{1.5cm}

        \Large
        im Sommersemester 2025

        \vspace{2cm}

        \textbf{Vorgelegt von:}\\[0.3cm]
        \normalsize
        Linus Breitenberger\\
        lb205@hdm-stuttgart.de\\
        Matrikelnummer: 43789

        \vfill

        \includegraphics[width=0.35\textwidth]{hdm_logo}\\[0.5cm]
        Hochschule der Medien Stuttgart

        \vspace{0.8cm}
        \today

    \end{center}
\end{titlepage}

	% \input{ads/restriction}
	% \input{ads/declaration}
	\newpage
	
	\tableofcontents
	\newpage

	\pagenumbering{arabic} % arabic pagenumbers

	% chapters:
	\section{Einleitung}
\label{chap:einleitung}

\subsection{Motivation und Projektziele}
Das vorliegende Projekt entstand im Rahmen des Moduls „Information Retrieval“ und diente der praktischen Anwendung der im Kurs vermittelten theoretischen Grundlagen. Die zentrale Aufgabenstellung bestand darin, eine eigenständige Suchanwendung auf Basis der etablierten Open-Source-Technologie Apache Solr zu konzipieren und umzusetzen.

Für die Umsetzung wurde die Pokémon API (pokeAPI.co) als Datenquelle ausgewählt. Auf dieser Grundlage entstand die Anwendung „Solr Pokédex“. Im Rahmen des Projekts wurden sowohl übergeordnete Ziele als auch konkrete funktionale Anforderungen definiert. Ein zentrales Ziel war die Entwicklung einer vollständigen Indexierungspipeline, die in der Lage ist, Daten automatisiert von der Quelle abzurufen, zu bereinigen und für Solr entsprechend aufzubereiten. Darüber hinaus sollte ein robustes und erweiterbares Solr-Schema entworfen werden, das die Struktur und Eigenschaften des gewählten Datensatzes sinnvoll abbildet. Für eine realistische und aussagekräftige Suchumgebung war die Indexierung eines relevanten Datenbestands vorgesehen, der mindestens 1000 einzigartige Einträge umfasst. Abgerundet wurde das Projektziel durch die Entwicklung einer simplen Weboberfläche, welche eine einfache und benutzerfreundliche Interaktion mit der Suchmaschine ermöglichen sollte.

\subsection{Funktionale Kernanforderungen}

Zur Erreichung der genannten Ziele musste die Anwendung bestimmte funktionale Anforderungen erfüllen. Dazu gehörte die Unterstützung verschiedener Suchanfragetypen, darunter eine klassische Keywordsuche über zentrale Textfelder, eine Phrasensuche zur gezielten Suche nach exakten Wortfolgen, eine Wildcardsuche mit Platzhaltern für flexible Suchmuster sowie eine facettierte Suche, die das Filtern von Ergebnissen nach Kriterien wie Pokémon-Typ oder Generation ermöglicht.

Um die Benutzerfreundlichkeit weiter zu erhöhen, wurden außerdem Funktionen zur Fehlerbehandlung und Ähnlichkeitssuche integriert. So sollte das System in der Lage sein, bei fehlerhaften Eingaben passende Korrekturvorschläge zu liefern („Meinten Sie…?“) und zusätzlich thematisch verwandte Inhalte zu einem Suchergebnis anzuzeigen („More like this“).

Über die funktionalen Kernanforderungen hinaus wurden einige optionale Erweiterungen als sogenannte „Stretch Goals“ formuliert. Dazu zählte unter anderem eine Autocompletion-Funktion, die während der Eingabe bereits passende Suchvorschläge anbietet, sowie ein Highlighting-Mechanismus, der die gesuchten Begriffe direkt in der Ergebnisvorschau visuell hervorhebt.

\subsection{Vorstellung der Suchmaschine Solr Pokédex}
\label{sec:vorstellung_pokedex}

Die im Rahmen dieses Projekts entwickelte Anwendung trägt den Namen „Solr Pokédex“ und ist eine spezialisierte Suchmaschine für Pokémon. Sie bietet einen umfassenden Index, der alle 1025 Pokémon der Generationen eins bis neun abdeckt.

Jedes Pokémon wird als eigenständiges Dokument in Apache Solr gespeichert und mit einer Vielzahl von detaillierten Metadaten angereichert. Diese Daten wurden sorgfältig ausgewählt, um sowohl eine gezielte Suche nach Fakten als auch eine explorative Volltextsuche zu ermöglichen. Zu den zentralen indexierten Feldern gehören:
\begin{itemize}
    \item \textbf{Stammdaten:} Name, Pokédex-ID, Typ(en), Generation, Größe und Gewicht.
    \item \textbf{Fähigkeiten und Kampfwerte:} Alle erlernbaren Fähigkeiten sowie die Basiswerte für Lebenspunkte (HP), Angriff, Verteidigung etc.
    \item \textbf{Beschreibender Text:} Ein separates Feld namens \texttt{flavor\_text} enthält die offiziellen Beschreibungen aus den Spielen. Mit seinem größeren Textumfang bildet dieses Feld die ideale Grundlage für eine freie Volltextsuche, die über die Suche nach reinen Fakten hinausgeht.
\end{itemize}

Die Interaktion mit der Suchmaschine erfolgt über eine mit Flask entwickelte Weboberfläche, die unter \texttt{http://localhost:5000} erreichbar ist. Die gesamte Anwendung ist containerisiert und lässt sich mittels Docker Compose und einem bereitgestellten Installationsskript (\texttt{install.sh}) unkompliziert einrichten und starten.

\subsection{Verwendeter Technologie-Stack}
\label{sec:tech_stack}

Die Architektur des „Solr Pokédex“ basiert auf einer Auswahl bewährter Open-Source-Technologien, die gezielt für ihre jeweilige Aufgabe im Projekt eingesetzt wurden. Der Stack lässt sich in die Bereiche Backend, Frontend, Datenquelle und Deployment unterteilen. Die prozentuale Verteilung der Programmiersprachen im Projekt (siehe Abbildung \ref{fig:language-distribution}) spiegelt diese Aufteilung wider.

\begin{figure}[h!]
\centering
\begin{tikzpicture}
\begin{axis}[
    xbar,
    width=12cm,
    height=6cm,
    xlabel={Anteil (\%)},
    symbolic y coords={Dockerfile,HTML,Shell,CSS,JavaScript,Python},
    ytick=data,
    nodes near coords,
    nodes near coords align={horizontal},
    xmin=0,
    xmax=50
]
\addplot coordinates {
    (0.9,Dockerfile)
    (8.0,HTML)
    (12.8,Shell)
    (15.4,CSS)
    (17.4,JavaScript)
    (45.5,Python)
};
\end{axis}
\end{tikzpicture}
\caption{Sprachverteilung im Codebestand}
\label{fig:language-distribution}
\end{figure}


\paragraph{Backend und Datenverarbeitung (Python, Apache Solr)}
Das Herzstück der Anwendung bildet das Backend, das primär in \textbf{Python} implementiert ist. Python wurde aufgrund seiner exzellenten Bibliotheken für Webentwicklung und Datenverarbeitung sowie seiner einfachen Lesbarkeit gewählt.
\begin{itemize}
    \item \textbf{Apache Solr:} Als Suchserver-Technologie wurde Solr eingesetzt, da es in der Lehrveranstaltung als Standard vorgegeben war. Eine freie Wahl zwischen verschiedenen Suchmaschinen bestand daher nicht, auch wenn Alternativen wie OpenSearch oder Elasticsearch ebenfalls interessante Optionen gewesen wären. Dennoch überzeugt Solr durch eine hohe Performance, eine flexible Schema-Definition und eine mächtige Query-Syntax, was es zu einer geeigneten Grundlage für die Indexierung und komplexe Abfrage der Pokémon-Daten macht. Die Konfigurationen des Solr-Cores befinden sich im Verzeichnis \texttt{solr/configsets/}.
    
    \item \textbf{Flask}: Das leichtgewichtige Web-Framework Flask dient als Brücke zwischen dem Frontend und dem Solr-Server. Im Rahmen der Lehrveranstaltung wurde eine einfache Basisanwendung auf Grundlage von Flask bereitgestellt, welche ich für meine Anwendung entsprechend angepasst und erweitert habe. Die Datei \texttt{web/web\_app.py} verarbeitet die HTTP-Anfragen der Benutzeroberfläche, konstruiert die entsprechenden Solr-Queries und gibt die Ergebnisse als gerenderte HTML-Seite zurück.
    
    \item \textbf{Datenakquise:}
\end{itemize}
Zu Beginn entwickelte ich ein einzelnes, umfassendes Skript namens \texttt{fetcher\_v2.py}, das alle erforderlichen Funktionen zur Befüllung der Suchmaschine in sich vereinte. Mit der Zeit und der Implementierung zusätzlicher Features wuchs dieses Skript jedoch kontinuierlich an, bis es schließlich unübersichtlich und schwer wartbar wurde. Aus diesem Grund entschied ich mich für eine Refaktorierung und teilte das ursprüngliche Skript in mehrere spezialisierte Module auf:

\begin{itemize} \item \texttt{main.py}: Das Hauptskript, das den gesamten Datenabfrage- und Indexierungsprozess orchestriert \item \texttt{api\_client.py}: Verwaltet die Kommunikation mit der Pokemon API
\item \texttt{data\_processor.py}: Verarbeitet und transformiert die Pokemon-Daten für die Indexierung \item \texttt{solr\_indexer.py}: Übernimmt das Setup des Solr-Schemas und die Dokumentenindexierung \item \texttt{config.py}: Enthält Konfigurationseinstellungen und das Logging-Setup \end{itemize}

Das ursprüngliche Skript \texttt{fetcher\_v2.py} fungierte als zentrale Komponente zur Befüllung der Suchmaschine und übernahm sämtliche Aufgaben von der Konfiguration des Solr-Schemas über den Abruf der Daten von der PokeAPI bis hin zu deren Verarbeitung und finaler Indexierung. Es war so konzipiert, dass es auch auf einem frischen Solr-Core ohne manuelle Vorbereitung funktionsfähig war. Die Daten wurden systematisch bereinigt, angereichert und in eine für Solr optimierte Struktur überführt. Ein integrierter Batching-Mechanismus und eine detaillierte Protokollierung gewährleisteten dabei sowohl Effizienz als auch Transparenz im gesamten Indexierungsprozess. Eine detailliertere Beschreibung der einzelnen Komponenten und deren Funktionsweise folgt in den nachstehenden Kapiteln.

\paragraph{Frontend (HTML, CSS, JavaScript)}
Die Benutzeroberfläche wurde mit klassischen Web-Technologien realisiert, um eine einfache und reaktionsschnelle User Experience zu gewährleisten.
\begin{itemize}
    \item \textbf{HTML und CSS:} Die Struktur der Webseite ist in der Template-Datei \texttt{web/templates/index.html} definiert. Das Styling erfolgt über eine separate CSS-Datei (\texttt{web/static/style.css}).
    
    \item \textbf{JavaScript:} Für die dynamische Interaktivität auf der Client-Seite kommt pures JavaScript \texttt{web/\allowbreak static/\allowbreak js/\allowbreak main.js} zum Einsatz. Eine zentrale Funktion ist die Darstellung der Detailansicht eines Pokémon. Bei einem Klick auf ein Suchergebnis wird kein neuer Seitenaufruf ausgelöst. Stattdessen wird ein modales Fenster (Modal) über die bestehende Seite gelegt, das die Detailinformationen des ausgewählten Pokémon anzeigt. Dieser Ansatz verbessert die Benutzererfahrung, da der Kontext der Suchergebnisse erhalten bleibt.
\end{itemize}

\paragraph{Datenquelle}
Als externe Datengrundlage dient die \textbf{Pokémon API (pokeAPI.co)}. Sie bietet eine umfangreiche und gut strukturierte Sammlung von Pokémon-Daten im JSON-Format, die sich ideal für die Verarbeitung und Indexierung eignete.

\paragraph{Deployment und Automatisierung (Docker, Shell)}
Um eine einfache und reproduzierbare Einrichtung der Anwendung zu garantieren, wurde auf Containerisierung und Skripting gesetzt.
\begin{itemize}
    \item \textbf{Docker und Docker Compose:} Die gesamte Anwendung, inklusive des Solr-Servers und der Flask-App, wird durch die Datei \texttt{docker-compose.yml} als Multi-Container-Anwendung definiert. Dies isoliert die Komponenten und vereinfacht das Deployment erheblich.
    
    \item \textbf{Shell-Skripting:} Das Skript \texttt{install.sh} automatisiert den gesamten Setup-Prozess: Es richtet die Python-Umgebung ein, installiert Abhängigkeiten, startet die Docker-Container und initiiert die erstmalige Datenindexierung.
\end{itemize}
	\section{Datengrundlage und Datenakquise}
\label{chap:daten}

\subsection{Die Datenquelle: Pokémon-API (pokeAPI.co)}
\label{sec:datenquelle}
Beschreibung der API, ihrer Struktur und warum sie für das Projekt geeignet ist.

\subsection{Analyse der Datenstruktur und relevanter Endpunkte}
\label{sec:datenstruktur}
Welche Endpunkte (z.B. /pokemon, /type) haben Sie genutzt? Wie sieht das JSON-Format aus?

\subsection{Datenakquise und Vorverarbeitung mit \texttt{fetcher\_v2.py}}
\label{sec:datenakquise}

\subsection{Automatisierter Datenabruf via Python-Skript}
Erläutern Sie die Funktionsweise des Skripts.

\subsection{Datenbereinigung und Transformation für die Indexierung}
Wie haben Sie die JSON-Daten aufbereitet, damit Solr sie verarbeiten kann? (z.B. Flachen von Strukturen, Auswahl relevanter Felder).

\subsection{Technische Besonderheiten: Implementierung eines Rate-Limits}
Erwähnen Sie, dass Sie die API-Richtlinien respektieren und wie Sie das technisch umgesetzt haben.


\section{Systemarchitektur und Konzeption}
\label{chap:architektur}

\subsection{Überblick der containerisierten Gesamtarchitektur}
\label{sec:gesamtarchitektur}
Fügen Sie hier ein Diagramm ein, das das Zusammenspiel von Docker, Flask-App und Solr-Container zeigt. Beschreiben Sie die Architektur.

\subsection{Entwurf des Solr-Indexschemas (\texttt{solr/configsets/})}
\label{sec:solr_schema}

\subsection{Definition zentraler Feldtypen}
Welche Feldtypen (z.B. text\_de, string, pint) haben Sie definiert und warum?

\subsection{Struktur des Index: Definierte Felder}
Listen Sie die wichtigsten Felder Ihres Index auf (z.B. \texttt{name}, \texttt{primary\_type}, \texttt{all\_abilities}, etc.) und beschreiben Sie deren Zweck.

\subsection{Nutzung von \texttt{copyField} für eine übergreifende Keywordsuche}
Erklären Sie, wie \texttt{copyField} funktioniert und warum Sie es für die einfache Suche eingesetzt haben.


\section{Implementierung der Kernkomponenten}
\label{chap:implementierung}

\subsection{Indexierungspipeline}
\label{sec:impl_indexing}
Beschreiben Sie den Prozess vom Start des Fetchers bis zum fertigen Dokument in Solr.

\subsection{Entwicklung der Webanwendung mit Flask (\texttt{web/web\_app.py})}
\label{sec:impl_flask}
Erläutern Sie die Struktur der Flask-App: Backend-Routen und Frontend-Templates. Fügen Sie hier Screenshots der UI ein.

\subsection{Implementierung der Suchfunktionalitäten}
\label{sec:impl_suche}
Gehen Sie auf die einzelnen Suchtypen ein, wie sie in der README beschrieben sind (Keyword, Phrase, Facetten, etc.) und wie Sie diese mit Solr-Anfragen umgesetzt haben.

\subsection{Automatisierung des Setups (\texttt{install.sh})}
\label{sec:impl_install_script}
Erklären Sie kurz den Zweck und die Funktionsweise des Installationsskripts.


\section{Evaluation und Optimierung}
\label{chap:evaluation}

\subsection{Funktionale Tests der implementierten Suchanfragetypen}
\label{sec:eval_tests}
Wie haben Sie sichergestellt, dass die Suche wie erwartet funktioniert? Zeigen Sie Beispiele.

\subsection{Bewertung und Optimierung des Relevanzrankings}
\label{sec:eval_relevanz}
Haben Sie das Ranking angepasst (z.B. durch Boosting von Feldern)? Diskutieren Sie die Ergebnisse.

\subsection{Diskussion der optionalen Features}
\label{sec:eval_optional}
Falls implementiert, beschreiben Sie hier Highlighting und Autocompletion. Falls nicht, erwähnen Sie, warum nicht und dass es Teil des Ausblicks ist.
  \section{Evaluation und Optimierung}

Die Validierung der implementierten Suchfunktionalitäten erfolgte mittels eines automatisierten Test-Scripts mit 48 verschiedenen Suchszenarien. Alle Tests liefen erfolgreich durch (Success Rate: 100\%), was zunächst einmal die technische Stabilität des Systems bestätigt.

\subsection{Funktionale Tests der Suchanfragetypen}

\subsubsection{Namensbasierte Suche}

Die exakte Namenssuche funktioniert tadellos -- bei Queries wie \texttt{pikachu} oder \texttt{charizard} erreicht das System perfekte Werte (Precision, Recall und F-Measure jeweils 1.0). Interessant wird es bei der partiellen Namenssuche mit Wildcards: Hier zeigt sich, dass Substring-Matching grundsätzlich gut funktioniert, wobei kürzere Fragmente wie \texttt{char} erwartungsgemäß mehr (und damit weniger präzise) Treffer liefern als spezifischere wie \texttt{saur}.

\begin{table}[h!]
\centering
\begin{tabular}{|l|c|c|c|}
\hline
\textbf{Suchtyp} & \textbf{Precision} & \textbf{Recall} & \textbf{F-Measure} \\
\hline
Exakte Namen & 1.00 & 1.00 & 1.00 \\
Partielle Namen & 0.71 & 1.00 & 0.80 \\
\hline
\end{tabular}
\caption{Performance der namensbasierten Suche}
\end{table}

\subsubsection{Thematische Suche}

Hier wird es problematisch: Die typ-basierte Suche ist praktisch defekt -- Suchanfragen nach \texttt{fire}, \texttt{water} oder \texttt{electric} liefern entweder gar keine oder völlig irrelevante Ergebnisse. Das ist ein kritischer Bug, der dringend behoben werden muss.

Die Fähigkeiten-Suche funktioniert zwar technisch, hat aber ein massives Präzisionsproblem: Bei Queries wie \texttt{overgrow} oder \texttt{blaze} werden zwar alle relevanten Pokémon gefunden (Recall = 1.0), aber gleichzeitig auch jede Menge irrelevante Treffer, was zu einer miserablen Precision von nur 0.15 führt.

\subsubsection{Interaktive Features}

Das Autocomplete arbeitet blitzschnell (8-9ms Antwortzeit) und liefert zu 100\% relevante Vorschläge -- hier läuft alles rund. Die Rechtschreibkorrektur hingegen existiert zwar, macht aber praktisch nichts: Tippfehler werden nicht korrigiert, obwohl das System nicht abstürzt.

\subsection{Information Retrieval Metriken}

Die Gesamtperformance des Systems zeigt ein gemischtes Bild:

\begin{table}[h!]
\centering
\begin{tabular}{|l|c|l|}
\hline
\textbf{Metrik} & \textbf{Wert} & \textbf{Bewertung} \\
\hline
Overall Precision & 0.44 & Ausbaufähig \\
Overall Recall & 0.93 & Sehr gut \\
F-Measure & 0.52 & Okay \\
Mean Reciprocal Rank & 1.00 & Perfekt \\
Mean Average Precision & 0.90 & Sehr gut \\
\hline
\end{tabular}
\caption{Gesamtperformance des Suchsystems}
\end{table}

Das System hat eine interessante Charakteristik: Es findet fast alles was relevant ist (hoher Recall), produziert dabei aber viel "Rauschen" in Form irrelevanter Treffer (niedrige Precision). Positiv ist, dass das erste Ergebnis praktisch immer das richtige ist (MRR = 1.0).

\subsection{Performance-Analyse}

Mit durchschnittlich 16ms Antwortzeit ist das System erfreulich schnell -- alle Anfragen werden in unter 20ms beantwortet. Hier gibt es definitiv keine Performance-Probleme.

\subsection{Identifizierte Optimierungsfelder}

Basierend auf den quantitativen Ergebnissen lassen sich klare Prioritäten ableiten:

\textbf{Identifizierte Verbesserungsfelder:}

\textbf{Kritische Probleme:}
\begin{itemize}
\item Typ-basierte Suche ist komplett defekt und liefert keine verwertbaren Ergebnisse
\item Overall Precision von 0.44 deutlich unter dem angestrebten Niveau von >0.6
\end{itemize}

\textbf{Relevante Schwächen:}
\begin{itemize}
\item Fähigkeiten-Suche hat mit 15\% Precision erhebliche Relevanzprobleme
\item Ranking-Qualität in den Top-10 Ergebnissen ist verbesserungswürdig
\end{itemize}

\textbf{Optimierungspotenzial:}
\begin{itemize}
\item Query-Expansion könnte thematische Suchen verbessern
\item Rechtschreibkorrektur funktioniert grundsätzlich, könnte aber erweitert werden
\end{itemize}

\subsection{Fazit der Evaluation}

Das entwickelte Suchsystem zeigt eine klare Zweiteilung in der Funktionalität: Während namensbasierte Suchen zufriedenstellende Ergebnisse liefern und durch gute Performance (16ms Antwortzeit) überzeugen, offenbart die systematische Evaluation mit standardisierten IR-Metriken erhebliche Defizite bei thematischen Suchfunktionen. 

Die dokumentierten Schwächen -- insbesondere die defekte Typ-Suche und die niedrige Overall Precision -- zeigen konkrete Ansatzpunkte für zukünftige Weiterentwicklungen auf. Die 100\%ige technische Stabilität und die funktionale Grundarchitektur demonstrieren die erfolgreiche Umsetzung der IR-Konzepte und bieten eine solide Basis für entsprechende Verbesserungen.
	\section{Fazit und Ausblick}
\label{chap:fazit}

\subsection{Zusammenfassung der Projektergebnisse}
\label{sec:fazit_zusammenfassung}

Die entwickelte Suchmaschine \texttt{Solr Pokédex} demonstriert erfolgreich die praktische Umsetzung von Information Retrieval-Konzepten in einer vollständigen Suchanwendung. Mit 1025 indexierten Pokémon aus neun Generationen und einer containerisierten Architektur wurde eine funktionsfähige Suchmaschine geschaffen, die verschiedene Suchmodi unterstützt -- von einfacher Keyword-Suche über facettierte Filter bis hin zu Autocomplete-Funktionalität.

Die technische Umsetzung überzeugt durch eine saubere Modularisierung der Komponenten: Von der systematischen Datenakquise über die PokeAPI bis zur responsiven Flask-Weboberfläche arbeiten alle Teile zuverlässig zusammen. Besonders positiv hervorzuheben ist die technische Stabilität und die exzellente Performance von 16ms durchschnittlicher Antwortzeit.

\subsection{Reflektion der Herausforderungen und Lösungsansätze}
\label{sec:fazit_herausforderungen}

Die größte technische Hürde stellte das Schema-Design dar – die Transformation verschachtelter JSON-Strukturen in ein suchoptimiertes Solr-Schema erforderte durchdachte Entscheidungen zwischen Flexibilität und Performance. Die implementierte Flattening-Strategie mit separaten Feldern für Typ-Kategorien und die Nutzung von Copy-Fields für Spell-Check haben sich bewährt.

Die Containerisierung mit Docker Compose eliminierte Deployment-Probleme und machte das System plattformübergreifend nutzbar.

Die Evaluation offenbarte jedoch kritische Schwächen: Die typ-basierte Suche ist praktisch unbrauchbar und die Overall Precision von 0.44 liegt deutlich unter professionellen Standards. Diese Probleme resultieren hauptsächlich aus suboptimaler Query-Konfiguration im edismax-Parser und fehlender thematischer Gewichtung.

\subsection{Mögliche Erweiterungen und zukünftige Optimierungen}
\label{sec:fazit_ausblick}

Die identifizierten Performance-Probleme bieten konkrete Ansatzpunkte für Verbesserungen: Eine Überarbeitung der Feldgewichtung im edismax-Parser könnte die Precision erheblich steigern, während Query-Expansion-Techniken thematische Suchen verbessern würden.

Interessante Erweiterungen umfassen die Integration weiterer Datenquellen wie Movesets oder Zuchtinformationen, die Implementierung von Benutzer-Accounts mit personalisierten Favoriten und erweiterte Visualisierungen wie Statistik-Vergleiche oder Typ-Effektivitäts-Charts.

Technisch wäre eine Migration zu moderneren Suchmaschinen wie OpenSearch oder die Integration von Machine Learning für intelligentere Ranking-Algorithmen denkbar. Die solide Architektur-Basis ermöglicht solche Erweiterungen ohne grundlegende Systemänderungen.

Das Projekt zeigt exemplarisch, wie theoretische IR-Konzepte in praktische Anwendungen überführt werden können – mit allen Erfolgen und Lernmöglichkeiten, die ein reales System mit sich bringt.


% --- Literaturverzeichnis ---
% Falls Sie BibTeX/BibLaTeX verwenden, kommt hier der Befehl zum Drucken hin.
% Beispiel für BibLaTeX:
% \printbibliography[heading=bibintoc, title={Literatur- und Quellenverzeichnis}]


% --- Anhang ---
\appendix % Schaltet auf Anhangs-Modus (A, B, C...)
	
	\appendix
	  \newpage
	  \listoffigures
	  \newpage
	  % \listoftables
	  % \input{ads/appendix-figures}

	\printglossary[type=\acronymtype]
	\printglossary[toctitle=Glossar]
	\lstlistoflistings
	
	% Print bibliography using BibLaTeX (replaces old \bibliography command)
	\printbibliography[heading=bibintoc,title=References]

\end{document}
